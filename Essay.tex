\documentclass{article}
\usepackage[utf8]{inputenc}
\usepackage[english,russian]{babel}
\usepackage{amsmath}
\usepackage[argument]{graphicx}


\title{Преобразование базиса Грёбнера нульмерного идеала к иному мономиальноальному упорядочению.}
\author{Федоров Глеб M33351}
\date{Октябрь 2020}

\begin{document}
    \begin{titlepage}
        \maketitle
    \end{titlepage}
    
\section{Оглавление}
    \subsection{Постановка проблемы}
    \subsection{Дополнительная теория}
    \subsection{Алгоритм FGLM}
    \subsection{Примерная архитектура программы}
    \newpage
\section{Постановка проблемы}
    Дан базис Грёбнера нульмерного идеала, построенный на мономиальном упорядочении $m1$.
    Привести данный базис к иному мономиальному упорядочению $m2$.
    \newpage
\section{Дополнительная теория} 
    \subsection{Исключающий идеал}
    Теорема(об исключении): Пусть $I \subset F[x_1, x_2,\ldots, x_s]$ - идеал и G - его базис Грёбнера по отношению
    к lex-упорядочению с $x_1>x_2>\ldots>x_n$. Тогда $\forall l : 0 \leq l \leq n $ множество
    $$G_l = G \cap F[x_1, x_2,\ldots,x_s]$$ является базисом Грёбнера $l$-го исключающего идеала $I_l$.
    
    Доказательство: 

    \subsection{Нульмерный идеал}
       Теорема: Пусть поле $F$ алгебраически замкнуто и $I\in F[x_1, x_2,\ldots, x_n]$.
       Тогда следующие условия эквивалентны:
        
        \begin{enumerate}
            \item Алгебра $A=F[x_1, x_2,\ldots, x_n]$ - I конечномерна над F.
            \item $V(I) \subset F^n$ конечно.
            \item Если $G$ - базис Грёбнера идеала $I$, то $$\forall i \exists m_i \geq 0 : x_i^{m_i} = LM(g)$$ для некоторого $g \in G$.
            \item Для каждой переменной $x_i$ исключающий идеал $I \cap F[x_1, x_2,\ldots, x_n]$ является ненулевым.
        \end{enumerate}

      Доказательство:
      Идеал, удовлетворяюзий данной теореме называется нульмерным

\newpage
\section{Алгоритм FGLM}

\end{document}
