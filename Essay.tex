\documentclass{article}
\usepackage[utf8]{inputenc}
\usepackage[english,russian]{babel}
\usepackage{amsmath}
\usepackage[argument]{graphicx}
\usepackage{xcolor}
\usepackage{hyperref}
\usepackage{indentfirst}

\definecolor{linkcolor}{HTML}{799B03} % цвет ссылок
\definecolor{urlcolor}{HTML}{799B03} % цвет гиперссылок
 
\hypersetup{pdfstartview=FitH,  linkcolor=linkcolor,urlcolor=urlcolor, colorlinks=true}


\title{Преобразование базиса Грёбнера нульмерного идеала к иному мономиальноальному упорядочению.}
\author{Федоров Глеб M33351}
\date{Октябрь 2020}

\begin{document}
    \begin{titlepage}
        \maketitle
    \end{titlepage}
    
\section{Оглавление}
    \subsection{Постановка проблемы}
    \subsection{Дополнительная теория}
    \subsection{Алгоритм FGLM}
    \subsection{Примерная архитектура программы}
    \subsection{Используемая литература}
    \newpage
\section{Постановка проблемы}
    Дан базис Грёбнера нульмерного идеала, построенный на мономиальном упорядочении $m_1$.
    Привести данный базис к иному мономиальному упорядочению $m_2$.
    \newpage
\section{Дополнительная теория} 
    \subsection{Исключающий идеал}

    \textbf{Определение:} Пусть дан $I = <f_1, \ldots, f_n> \in F[x_1,\ldots,x_n]$. Тогда l-м исключающим идеалом $I_l$ называется 
    идеал в $F[x_{l+1},\ldots, x_n]$, равный $I \cup F[x_{l+1},\ldots, x_n]$.

    \textbf{Теорема(об исключении):} Пусть $I \subset F[x_1, x_2,\ldots, x_s]$ - идеал и G - его базис Грёбнера по отношению
    к lex-упорядочению с $x_1>x_2>\ldots>x_n$. Тогда $\forall l : 0 \leq l \leq n $ множество
    $$G_l = G \cap F[x_1, x_2,\ldots,x_s]$$ является базисом Грёбнера $l$-го исключающего идеала $I_l$.
    
    \textbf{Доказательство:} 
     Зафиксируем $l$ в интервале $(0, n)$. Так как $G_l \in I_l$ по построению, достаточно показать, что 
     $$<LT(I_l)>=<LT(G_l)>$$ (по определению базиса Грёбнера). 
     Включение в одну сторону очевидно($<LT(G_l)> \in <LT(I_l)>$ по построению). Докажем, что $<LT(I_l)> \in <LT(G_l)>$.
     Для этого достаточно показать что $LT(f)$, где $f \in I_l$, делится на некоторый $g \in G_l$. 
     Заметим, что $f \in I$, то есть $LT(f)$ делится на $LT(g)$ для некоторго $g$(т.к. $G$ является базисом Грёбнера иделала $I$). 
     Так как $f \in I_l$, то $LT(g)$ содержит только переменные $x_{l+1}, \ldots, x_n$. Так как используется lex-упорядочении с 
     $x_1>x_2>\ldots>x_n$, то любой моном, содержащий хотя бы одну из переменных $x_1, \ldots, x_l$, больше всех мономов из
     $F[x_{x+1},\ldots, x_n$. Значит $g\in G_l$, что и требовалось доказать.
    \newpage

    \subsection{Соответсвие иделала и многообразия}

    \textbf{Определение:} Пусть $I \in F[x_1,\ldots, x_n]$ - некоторый идеал. Положим
    $$V(I) = {(a_1, \ldots, a_n) \in F^n : f(a_1, \ldots, a_n) = 0 \forall f \in I}$$

    \textbf{Теорема:} $V(I)$ является аффинным многообразием. В частности, если $I = <f_1, \ldots, f_n>$,
    то $V(I) = V(f_1,\ldots,f_n)$

    \textbf{Доказательство:} По теореме Гильберта о базисе идеал $I$ конечно
     порождён, $I = <f_1, \ldots, f_n>$. Покажем, что $V(I) = V(f_1,\ldots,f_n)$.
    Если $f(a_1,\ldots,a_n) = 0$ для всех полиномов $f \in I$, то $f_i(a_1,\ldots,a_n)=0$ (так как $f_i \in I$). Следовательно,
    $V(I) \in V(f_1,\ldots,f_n)$. С другой стороны, пусть $(a_1,\ldots,a_n) \in V(f_1,\ldots, f_n)$, и пусть $f\in I$.
    Так как $I = <f_1,\ldots, f_n>$, то $$f = \sum_{i=1}^s h_if_i$$ для некоторых $h_i \in F[x_1, \ldots, x_n]$.
    Но тогда 
    $$f(a_1,\ldots,a_n)=\sum_{i=1}^s h_i(a_1,\ldots,a_n)f_i(a_1,\ldots,a_n)=\sum_{i=1}^s h_i(a_1,\ldots,a_n)*0 = 0$$
    Следовательно, $V(f_1, \ldots, f_n) \in V(I)$, а значит эти два идеала равны. 

    \textbf{Следствие:} Многообразия определены идеалами.

    \newpage

    \subsection{Нульмерный идеал}

    \textbf{Теорема:} Пусть поле $F$ алгебраически замкнуто и $I\in F[x_1, x_2,\ldots, x_n]$.
       Тогда следующие условия эквивалентны:
        
        \begin{enumerate}
            \item Алгебра $A=F[x_1, x_2,\ldots, x_n]$ - I конечномерна над F.
            \item $V(I) \subset F^n$ конечно.
            \item Если $G$ - базис Грёбнера идеала $I$, то $$\forall i \exists m_i \geq 0 : x_i^{m_i} = LM(g)$$ для некоторого $g \in G$.
            \item Для каждой переменной $x_i$ исключающий идеал $I \cap F[x_1, x_2,\ldots, x_n]$ является ненулевым.
        \end{enumerate}

        \textbf{Доказательство:}
      Идеал, удовлетворяюзий данной теореме называется нульмерным

\newpage
\section{Алгоритм FGLM}
\newpage
\section{Используемая литература}
\begin{enumerate}
    \item \href{http://halgebra.math.msu.su/groebner.pdf}{http://halgebra.math.msu.su/groebner.pdf}
    \item Кокс Д., Литтл Дж., О'Ши Д. Идеалы, многообразия, кольца. Стр. 108. Стр. 153. 
    \item \href{https://www.math.lsu.edu/system/files/Groeb_presentation_final.pdf}{https://www.math.lsu.edu/system/files/Groeb_presentation_final.pdf}
\end{enumerate}


\end{document}
